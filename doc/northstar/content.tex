\section{Увод}

\subsection{Намена документа и циљна група}
Текст који следи описује намену апликације, функционалности које она пружа, захтеве који су постављени и идеје за даље унапређивање. Документ је намењен члановима развојног тима и клијенту, како би се дефинисала функционалност коју треба  реализовати.

\subsection{Резиме}
Пројекат “GravityCrusher” је део практичне наставе предмета Принципи софтверског инжењерства. У питању је видео игра која се игра у претраживачу, у којој два играча играју један против другог, у окружењу свемира, са реалистичним утицајем гравитације околних објеката.

\section{Циљ пројекта}
Циљ овог пројекта је израда игре која се извршава на серверу, а контролише и приказује на клијентима - Интернет претраживачима. Имплементација би обухватала израду веб сајта који би пружао основне функционалности као што су давање основних информација о производу и могућност пријаве корисника. Други део имплементације би обухватао израду севера за игру са којим би клијенти комуницирали током играња игре и који би био задужен за одржаванје стања игре, примање команди од клијената и константно прослеђиванје стања игре клијентима који то стање приказују на екран. Веб сајт коме корисник приступа би садржао и javascript апликацију чији задатак би био да прима команде од корисника и прослеђује их серверу, као и да прима стање игре од сервера и приказује га у графичком прозору на страници.

\section{Категорије корисника}
Апликација би имала три главне категорија корисника: Играче, који активно учествују у игрању игре у соби на серверу, Посматраче који могу да се прикључе соби и гледају играче како играју и Администраторе који могу да надгледају поруке у собама.

\subsection{Играч}
Играч је корисник веб сајта који учествује у игри задавањем команди једном од свемирских бродова. Играч може бити улоговани члан, или анониман, у ком случају није могуће чување његове статистике. Играч би комуницирао са игром на један од два предложена начина. Први начин је контрола помоћу тастатуре и/или миша како би се контролисао свемирски брод. Други начин би био помоћу команди које се налазе на веб страници поред прозора који графички приказује стање игре. Ове команде би се користиле подешавањем клизача или уношењем бројних вредности у назначена поља.


\subsection{Посматрач}
Посматрач је посетилац веб сајта који не учествује активно у игри, него у изабраној соби прати стање игре која се тренутно игра на серверу. Може бити улогован у свој профил или анонимни посетилац сајта.

\subsection{Администратор}
Администратор је посетилац који има могућност да ограничи коришћење одређених функција и забрани приступ  веб страници одређеним корисницима (из редова играча и редова посматрача) на одређени временски период. Његов задатак је да не допушта играчима да варају, као ни посматрачима да објављују неадекватан садржај у чету унутар соба за посматрање. Овакве кориснике треба да санкционише на адекватан начин. Остали играчи имају могућност да пријаве уочене неправилности администратору. Уколико неко од корисника злоупотреби ову могућност и често пријављује неадекватне проблеме за неправилности, администратор има право и такве кориснике да санкционише.

\section{Опис производа}
Све се одвија на сајту на који је могуће пријавити се. Постоје странице са описом апликације, o развојном тиму, страница за иницијализацију игре, страница са листом активних игара, као и страница са интерфејсом за игру.
Страница за иницијализацију игре садржи подешавања за именовање собе, подешавање противника, одабир мапе за игру.
Страница са листом активних игара омогућава одабирање и придруживање већ направљеној соби као гледалац или играч уколико је отворена за било кога.
Страница са интерфејсом за игру садржи основне информације о игри, неке контроле за игру, прозор за графичко представљање стања игре и скрипту за комуникацију са сервером и рендеровање графике.

\subsection{Опис архитектуре система}
Систем ће бити заснован на трослојној архитектури. На презентационом слоју система приказиваће се стање игре, кориснички интерфејс за одговарајућу категорију посетиоца сајта и остала обавештења, менији, упитници, анимације. На слоју пословне логике обрађиваће се подаци добијени од целокупног корисничког интерфејса презентационог слоја: ажурираће се стање игре, израчунаваће се и памтити релевантни статистички подаци за пријављене играче, одржаваће се ранг-листе, спајаће се удаљени играчи приликом иницијализовања нове игре итд. У слоју података налазиће се релациона база података.






\subsection{Преглед коришћених технологија}
Приликом имплементације система биће коришћене следеће технологије:
\begin{itemize}
    \item Презентациони слој: HTML5, CSS3, JavaScript, Bootstrap
    \item Слој пословне логике: Node.js, Sockets.io, SQL
    \item Слој података: MySQL Server
\end{itemize}

\section{Функционални захтеви} 
У овом одељку описују се функционалности које систем треба да пружи корисницима, и то заједничке и за саваку категорију понаособ.

\subsection{Заједничке функционалности за све типове корисника}

\subsubsection{Креирање новог налога}
Страница на сајту на којој је могуће унети своје корисничке податке и направити нови налог. Налог се потврђује преко e-mail поруке.

\subsubsection{Пријава на систем}
Страница на сајту на којој је могуће унети своје креденцијале, након чега је корисник аутентификован. Након тога статистика игара које корисник игра се везује са његовим налогом. Такође администратори добијају приступ својим могућностима након пријаве на систем.

\subsubsection{Претрага постојећих соба}
Страница на сајту на којој постоји листа постојећих соба, на којој је могуће прикључити се соби као играч уколико је играч потребан или као посматрач.

\subsubsection{Креирање нове собе}
Страница на сајту на којој се уносе жељена подешавања за нову собу. Подешавања могу бити име собе, мапа за игру, филтери за противника, могућност за четовање током игре и специфична подешавања за игру. Собе могу бити јавног типа, при чему је сваком кориснику дозвољено да се прикључи, односно приватне, при чему се позивајз само конкретни корисници да играју или посматрају игру.

\subsubsection{Размена порука унутар собе}
Омогућити систем за размену порука унутар сваке собе. Он би био активан уколико је такво подешавање собе.

\subsubsection{Примање стања игре и графичко приказивање}
Након уласка у собу у којој је игра у току, играчи и посматрачи покрећу скрипту која одржава конекцију са сервером за игру, од њега прима стање у игри у реалном времену. Након сваког примљеног пакета, скрипта генерише графички приказ стања и приказује га на екрану.

\subsection{Функционалности играча}

\subsubsection{Слање команди играча серверу}
Током трајања игре, играчи покрећу скрипту која одржава конекцију са сервером за игру и њему шаље улаз корисника у реалном времену.

\subsection{Функционалности администратора}

\subsubsection{Санкционисање играча}
Администратори имају могућности да санкционишу играче за слање некоректних порука тако што притисну на дугме поред имена играча у простору за размену порука, што удаљава корсиника из собе.

\section{Претпоставке и ограничења}
За пријаву је потребно да корисник поседује e-mail адресу како би било могуће потврдити валидност налога.
Кашњење пакета које резмењују сервер и клијенти може негативно утицати на искуство играња игре, тако да је неопходно дизајнирати игру тако да искуство није превише нарушено појавом малог кашњења у интернет вези.

\section{Квалитет}
Потребно је извршити тестирање исправног функционисања сајта и сервера за игру, као и брзине размене информација између клијената и сервера. Обратити пажњу на спречавање уноса малициозног SQL кода који би могао да оштети базу.

\section{Нефункционални захтеви}
Сервер мора да подржава покретање Node.js програма и MySQL базе података. Потребно је обезбедити компатибилност и исправно приказивање страница на свим релевантним претраживачима. Претраживачи морају подржавати HTML5 стандард који је неопходан за извршавање корисничких скрипти за графичко приказивање игре.

\section{Захтеви за корисничком документацијом}
Треба обезбедити страницу на сајту на којој би постојала упутства за коришћење сајта и играње игре за све врсте корисника.

\section{План и приоритети}

\subsection{Примарно}
Примарно је потребно израдити следеће делове производа:
\begin{itemize}

\item Дизајн веб сајта

\item Систем за пријаву корисника на систем

\item Сервер за игру који може да комуницира са клијентима и извршава више инстанци игре у реалном времену

\item Дизајн контрола за игру

\item Графички дизајн игре који ће бити приказан играчима и посматрачима током игре

\end{itemize}

\subsection{Додатно}
Поред основних функционалности, требало би обезбедити и:
\begin{itemize}
    
    \item Систем за чување статистике играча и неколико начина за рангирање

    \item Могућност размене порука унутар собе између играча и посматрача

\item Оптимизација сајта тако да је игру могуће играти на рачунару и телефону    
\end{itemize}
