\section{Увод}

\subsection{Резиме}
Дефинисање сценарија употребе у ситуацији када корисник жели да се пријави на свој налог, али
се не сећа лозинке.

\subsection{Намена документа и циљне групе}
Документ ће користити сви чланови пројектног тима у развоју пројекта, а може се 
користити и при писању упутства за употребу.

\subsection{Референце}
\begin{enumerate}
	\item Опис пројектног задатка
	\item Упутство за писање спецификације сценарија случаја употребе
\end{enumerate}

\subsection{Отворена питања}
\begin{table}[h!]
\centering
	
	\begin{tabu}{ || X[l] | X[l] | X[l] | X[l] || }
	\hline
	\textbf{Верзија} & \textbf{Датум} & \textbf{Кратак опис} & \textbf{Аутор} \\
	\hline
	\hline
	& & &\\
	\hline
	& & &\\
	\hline
	& & &\\
	\hline
	& & &\\
	\hline
	\end{tabu}
	\caption{Преглед отворених питања}
	\label{table:2}
		
\end{table}



\section{Подношење захтева за обнављањем лозинке}

\subsection{Кратак опис}
Овде је представљен сценарио тражења нове лозинке од система. Очекујемо да ће корисници ову
могућност користити уколико забораве своју лозинку.

\subsection{Ток догађаја}
\begin{enumerate}
	\item Корисник притиска дугме за обнављање лозинке
	\item Систем кориснику приказује посебну страницу на којој корисник уноси e-mail
		  адресу за налог за који жели да промени лозинку
	\item Систем кориснику шаље линк ка страници за унос нове лозинке на e-mail адресу коју је
	      унео приликом креирања налога
	\item Корисник уноси нову лозинку
	\item Корисник притиска дугме за потврду нове лозинке
	\item Систем исписује поруку кориснику да је промена лозинке успела
\end{enumerate}

\subsubsection{Не постоји налог за унету e-mail адресу}
\begin{enumerate}[label=2.\arabic*]
	\item Систем исписује поруку да не постоји кориснички налог за унету e-mail адресу
	\item Систем чисти поље за унос e-mail адресе
	\item Повратак на тачку 2
\end{enumerate}


\subsection{Посебни захтеви}
Нема.

\subsection{Предуслови}
Постоји налог са e-mail адресом за коју се покушава обнављање лозинке.

\subsection{Последице}
Корисник има нову лозинку и користи је уместо старе.
