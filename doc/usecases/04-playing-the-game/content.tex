\section{Увод}

\subsection{Резиме}
Дефинисање сценарија употребe током учествовања у игри.

\subsection{Намена документа и циљне групе}
Документ ће користити сви чланови пројектног тима у развоју пројекта, а може се 
користити и при писању упутства за употребу.

\subsection{Референце}
\begin{enumerate}
	\item Опис пројектног задатка
	\item Упутство за писање спецификације сценарија случаја употребе
\end{enumerate}

\subsection{Отворена питања}
\begin{table}[h!]
\centering
\small
	
	\begin{tabu}{ || X[l] | X[l] | X[l] | X[l] || }
	\hline
	\textbf{Верзија} & \textbf{Датум} & \textbf{Кратак опис} & \textbf{Аутор} \\
	\hline
	\hline
	1.0 & 10.03.2018. & Још увек се разматра како ће сама игра тачно изгледати. Потребно
	 је завршити имплементацију прототоипа игре, како би он могао да буде анализиран пре
	 доношења даљих одлука о садржају игре. & Никола Јовановић \\
	\hline
	& & &\\
	\hline
	& & &\\
	\hline
	& & &\\
	\hline
	\end{tabu}
	\caption{Преглед отворених питања}
	\label{table:2}
		
\end{table}



\section{Сценарио Играње игре}

\subsection{Кратак опис}
У питању је функционалност учествовања у игри, која обухвата активности које корисцник може да предузме унутар игре, као и информације које се кориснику презентују.

\subsection{Ток догађаја}
У наставку биће приказан главни (успешни) сценарио, као и престали алтернативни токови. Сценарио ће имати јединствени ток, при чему увек може бити прекинут, уколико корисник изађе са странице за играње игре. Ток догађаја се понавља циклично све до завршетка или прекида игре.

\begin{enumerate}
    \item Уколико је корисник играч који је направио собу, он чека да се противник придружи игри
    \item Почиње одбројавање од десет секунди до аутоматског почетка игре
    \item Опционо, корисник притиском на тастер 's' обавештава систем и другог играча да је 
         спремам за почетак игре и пре завршетка одбројавања
    \item Почетак игре, кориснику се приказује мапа на којој се налази зелени корисников брод на дну, 
	црвени противнички брод на врху и планете на средини прозора, као и бројач времена које
	је остало до краја партије између имена играча, посматрачи игру посматрају из угла 
	корисника који је направио собу
    \item Корисник задаје једну или више акција помоћу тастатуре
	\begin{itemize}
	    \item 'a': померање брода у лево
	    \item 'd': померање брода у десно
	    \item 'j': окретање брода око своје осе у лево
	    \item 'l': окретање брода око своје осе у десно
	    \item 's': испаљивање слабијег и бржег метка
	    \item 'k': испаљивање јачег али споријег метка
	\end{itemize}
    \item Систем ажурира стање игре и графички га приказује на екрану свим корисницима
          у соби
    \item Повратак на корак 5
    \item Игра је завршена, на екрану се приказује ко је победио, а ко изгубио
    \item Опционо, корисник притиском на дугме за реванш (\textit{rematch}) обавештава систем и другог 
	играча да би желео реванш
\item Сви корисници излазе из странице собе кликом на дугме за повратак (\textit{return to home})
    \item Соба се укида
\end{enumerate}

\subsubsection{Корисник притиска на дугме за укудање собе (\textit{close room})}
\begin{enumerate}[label=1.\arabic*]
	\item Систем укида собу коју је корисник направио
	\item Корисник се пребацује на почетну страницу
\end{enumerate}

\subsubsection{Корисник након уласка противника у собу притиска дугме за предају (\textit{surrender})}
\begin{enumerate}[label=2.\arabic*]
	\item Противнику се додељује победа
	\item Прелази се на стање 8
\end{enumerate}

\subsubsection{Противник такође обавештава да је спреман}
\begin{enumerate}[label=3.\arabic*]
	\item Прелазак на стање 4, пре истека одбројавања
\end{enumerate}

\subsubsection{Систем је детектовао да је неко од играча изгубио све животе или да је истекло време}
\begin{enumerate}[label=6.\arabic*]
	\item Победа се додељује другом играчу који није изгубио све животе, 
	    односно никоме ако је истекло време
	\item Прелазак на стање 8
\end{enumerate}

\subsubsection{Противник такође жели реванш}
\begin{enumerate}[label=9.\arabic*]
	\item Повратак на стање 4, почетак нове игре
\end{enumerate}

\subsection{Посебни захтеви}
Нема.

\subsection{Предуслови}
Игра мора да буде започета и кориснику мора да буде дозвољен приступ соби од стране система.

\subsection{Последице}
Чување опште статистике игре, као и статистике играча који су били пријављени на систем током игре.
