\section{Увод}

\subsection{Резиме}
Дефинисање сценарија употребe у случају жеље корисника да промени нешто од података на свом налогу.

\subsection{Намена документа и циљне групе}
Документ ће користити сви чланови пројектног тима у развоју пројекта, а може се 
користити и при писању упутства за употребу.

\subsection{Референце}
\begin{enumerate}
	\item Опис пројектног задатка
	\item Упутство за писање спецификације сценарија случаја употребе
\end{enumerate}

\subsection{Отворена питања}
\begin{table}[h!]
\centering
\small
	
	\begin{tabu}{ || X[l] | X[l] | X[l] | X[l] || }
	\hline
	\textbf{Верзија} & \textbf{Датум} & \textbf{Кратак опис} & \textbf{Аутор} \\
	\hline
	\hline
	1.0 & 10.03.2018. & Треба прецизирати да ли се ова опција приказује као могућност за
	   преглед профила са подацима које је могуће изменити или је ово потпуно одвојена
	   функционалност система. & Филип Мандић \\
	\hline
	& & &\\
	\hline
	& & &\\
	\hline
	& & &\\
	\hline
	\end{tabu}
	\caption{Преглед отворених питања}
	\label{table:2}
		
\end{table}



\section{Сценарио подешавања корисничког налога}

\subsection{Кратак опис}
Овде је представљен сценарио случаја коришћења система за подешавања корисничког налога.
Сценарио ће имати јединствени ток, при чему увек може бити прекинут, уколико корисник изађе из
странице. Могиће је изменити корисничко име и лозинку.

\subsection{Ток догађаја}
\begin{enumerate}
	\item Корисник уноси промене које жели да се остваре на његовом налогу
	\item Корисник притиска дугме за потврду тих промена
	\item Систем проверава да ли се лозинке у оба поља поклапају
	\item Систем проверава коректност података
	\item Систем исписује поруку о успешно извршеним променама
\end{enumerate}

\subsubsection{Лозинке у пољима се разликују}
\begin{enumerate}[label=4.\arabic*]
	\item Систем исписује поруку о грешци
	\item Систем уписује стару лозинку у оба поља
	\item Повратак на корак 1 главног сценарија
\end{enumerate}

\subsubsection{Ново корисничко име већ постоји у систему}
\begin{enumerate}[label=5.\arabic*]
	\item Систем исписује поруку о грешци
	\item Систем рестаурира старе податке чија промена је покушана
	\item Повратак на корак 1 главног сценарија.
\end{enumerate}

\subsection{Посебни захтеви}
Нема.

\subsection{Предуслови}
Корисник је пријављен на свој налог.

\subsection{Последице}
Корисник има налог са свим извршеним променама.
