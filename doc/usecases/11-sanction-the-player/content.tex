\section{Увод}

\subsection{Резиме}
Дефинисање сценарија употребе при санкционисању играча од стране администратора.

\subsection{Намена документа и циљне групе}
Документ ће користити сви чланови пројектног тима у развоју пројекта, а може се 
користити и при писању упутства за употребу.

\subsection{Референце}
\begin{enumerate}
	\item Опис пројектног задатка
	\item Упутство за писање спецификације сценарија случаја употребе
\end{enumerate}

\subsection{Отворена питања}
\begin{table}[h!]
\centering
	
	\begin{tabu}{ || X[l] | X[l] | X[l] | X[l] || }
	\hline
	\textbf{Верзија} & \textbf{Датум} & \textbf{Кратак опис} & \textbf{Аутор} \\
	\hline
	\hline
	& & &\\
	\hline
	& & &\\
	\hline
	& & &\\
	\hline
	& & &\\
	\hline
	\end{tabu}
	\caption{Преглед отворених питања}
	\label{table:2}
		
\end{table}



\section{Сценарио санкционисање играча}

\subsection{Кратак опис}
Посебна могућност администратора је да санкционише играче за које процени да се понашају
неприкладно приликом размене порука. Сакнционисани корисници се не могу пријавити на свој
налог у одређеном временском периоду.

\subsection{Ток догађаја}
\begin{enumerate}
	\item Администратор бира играча ког жели да санкционише
    \item Администратор бира на колико велики временски период да санкционише играча
	\item Систем удаљава санкционисаног играча из тренутне собе и одљављује га
	\item Санкционисаном кориснику се приказује почетна страна у режиму непријављеног
	      корисника
\end{enumerate}


\subsection{Посебни захтеви}
Нема.

\subsection{Предуслови}
Корисник који санкционише има администрторска права.
Корисник који је санкционисан је пријављен на свој налог.

\subsection{Последице}
Санкционисани корисник не може да се пријави на свој налог у одређеном временском периоду
након што је санкционисан.
