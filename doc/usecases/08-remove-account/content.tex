\section{Увод}

\subsection{Резиме}
Дефинисање сценарија употребе у случају жеље корисника да уклони свој налог.

\subsection{Намена документа и циљне групе}
Документ ће користити сви чланови пројектног тима у развоју пројекта, а може се 
користити и при писању упутства за употребу.

\subsection{Референце}
\begin{enumerate}
	\item Опис пројектног задатка
	\item Упутство за писање спецификације сценарија случаја употребе
\end{enumerate}

\subsection{Отворена питања}
\begin{table}[h!]
\centering
	
	\begin{tabu}{ || X[l] | X[l] | X[l] | X[l] || }
	\hline
	\textbf{Верзија} & \textbf{Датум} & \textbf{Кратак опис} & \textbf{Аутор} \\
	\hline
	\hline
	& & &\\
	\hline
	& & &\\
	\hline
	& & &\\
	\hline
	& & &\\
	\hline
	\end{tabu}
	\caption{Преглед отворених питања}
	\label{table:2}
		
\end{table}



\section{Сценарио уклањање корисничког налога}

\subsection{Кратак опис}
Овде је представљен сценарио случаја коришћења система за уклањање корисничког налога. Сценарио ће имати јединствени ток, при чему увек може бити прекинут, уколико корисник изађе из странице за креирање собе.

\subsection{Ток догађаја}
\begin{enumerate}
	\item Корисник притиска дугме за уклањање налога
	\item Корисник уноси своју шифру како би му било дозвољено да обрише налог
	\item Систем проверава шифру која је унета
	\item Систем исписује поруку о успешно уклоњеном налогу
\end{enumerate}

\subsubsection{Лозинка је погрешно унета}
\begin{enumerate}[label=4.\arabic*]
	\item Систем исписује поруку о грешци
	\item Систем чисти и премешта фокус на поље за унос лозинке
	\item Повратак на корак 2 главног сценарија
\end{enumerate}

\subsection{Посебни захтеви}
Нема.

\subsection{Предуслови}
Корисник је пријављен на свој налог.

\subsection{Последице}
Корисник је успешно уклонио свој налог.
Корисник је oдјављен и пребачен на почетну страну коју виде нерегистровани корисници.
Уклоњени налог се игнорише у целом систему до следеће пријаве корисника.
