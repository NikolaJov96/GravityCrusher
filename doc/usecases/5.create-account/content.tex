\section{Увод}

\subsection{Резиме}
Дефинисање сценарија употребе при креирању новог налога на сајту.

\subsection{Намена документа и циљне групе}
Документ ће користити сви чланови пројектног тима у развоју пројекта, а може се 
користити и при писању упутства за употребу.

\subsection{Референце}
\begin{enumerate}
	\item Опис пројектног задатка
	\item Упутство за писање спецификације сценарија случаја употребе
\end{enumerate}

\subsection{Отворена питања}
\begin{table}[h!]
\centering
	
	\begin{tabu}{ || X[l] | X[l] | X[l] | X[l] || }
	\hline
	\textbf{Верзија} & \textbf{Датум} & \textbf{Кратак опис} & \textbf{Аутор} \\
	\hline
	\hline
	& & &\\
	\hline
	& & &\\
	\hline
	& & &\\
	\hline
	& & &\\
	\hline
	\end{tabu}
	\caption{Преглед отворених питања}
	\label{table:2}
		
\end{table}



\section{Сценарио Креирање налога на сајту}

\subsection{Кратак опис}
Корисник прави налог на сајту. Овај корак није неопходан како би корисник играо саму
игру, али омогућава кориснику боље искуство нудећи неке додатне опције. Корисник ће кроз
посебну форму уносити информације неопходне за прављење налога. У те податке спада
корисничко име, e-mail адреса и лозинка.

\subsection{Ток догађаја}
Сценарио ће имати јединствени ток, при чему увек може бити прекинут, уколико корисник
изађе из странице за прављење налога.

\begin{enumerate}
    \item Корисник уноси корисничко име
    \item Систем проверава да ли је унето корисничко име јединствено у систему
    \item Корисник уноси своју e-mail адресу
    \item Корисник уноси лозинку у два поља одвојено
    \item Корисник притиска дугме за креирање налога
    \item Систем проверава да ли је при оба уноса лозинке унет исти текст
    \item Систем проверава да ли је лозинка довољно добра
    \item Систем исписује поруку да је налог успешно креиран
    \item Корисник активира налог активацијом линка из примљене e-mail поруке
\end{enumerate}

\subsubsection{Унето корисничко име већ постоји у систему}
\begin{enumerate}[label=2.\arabic*]
    \item Систем исписује поруку да треба покушати са другим именом
    \item Систем враћа курзор у поље за унос корисничког имена
    \item Повратак на корак 1 из главног сценарија
\end{enumerate}

\subsubsection{Унете су различите лозинке у два поља}
\begin{enumerate}[label=6.\arabic*]
    \item Систем исписује поруку о унетим различитим лабелама
    \item Систем чисти поља за унос лозинки и премешта курзор у прво
    \item Повратак на тачку четири
\end{enumerate}

\subsubsection{Унета лозинка није довољно добра}
\begin{enumerate}[label=7.\arabic*]
    \item Систем исписује поруку да лозинка није задовољавајућа
    \item Систем чисти поља за унос лозинки и премешта фокус на прво
    \item Повратак на тачку 4
\end{enumerate}

\subsection{Посебни захтеви}
Нема.

\subsection{Предуслови}
Учитавање странице на којој се уносе подаци поребни за регистрацију.
Корисник мора да поседује e-mail адресу да би могао да направи налог.

\subsection{Последице}
Корисник има ваљан налог на који може да се улогује. 
