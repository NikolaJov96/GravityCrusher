\section{Увод}

\subsection{Резиме}
Дефинисање сценарија употребе одјављивања са система.

\subsection{Намена документа и циљне групе}
Документ ће користити сви чланови пројектног тима у развоју пројекта, а може се 
користити и при писању упутства за употребу.

\subsection{Референце}
\begin{enumerate}
	\item Опис пројектног задатка
	\item Упутство за писање спецификације сценарија случаја употребе
\end{enumerate}

\subsection{Отворена питања}
\begin{table}[h!]
\centering
	
	\begin{tabu}{ || X[l] | X[l] | X[l] | X[l] || }
	\hline
	\textbf{Верзија} & \textbf{Датум} & \textbf{Кратак опис} & \textbf{Аутор} \\
	\hline
	\hline
	& & &\\
	\hline
	& & &\\
	\hline
	& & &\\
	\hline
	& & &\\
	\hline
	\end{tabu}
	\caption{Преглед отворених питања}
	\label{table:2}
		
\end{table}



\section{Сценарио санкционисање играча}

\subsection{Кратак опис}
Сценарио дефинише кораке који се предузимају приликом интеракције пријављеног корисника
и система у ситуацији у којој корисник жели да се одјави са система.

\subsection{Ток догађаја}
\begin{enumerate}
    \item Пријављен корисник бира дугме за одјављивање са система
    \item Систем одјављује корисника
    \item Систем преусмерава корисника на почетну страну
\end{enumerate}


\subsection{Посебни захтеви}
Нема.

\subsection{Предуслови}
Корисник мора бити пријављен на систем.

\subsection{Последице}
Корисник није пријављен на систем.

