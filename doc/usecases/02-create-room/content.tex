\section{Увод}

\subsection{Резиме}
Дефинисање сценарија употребе при креирању нове собе.

\subsection{Намена документа и циљне групе}
Документ ће користити сви чланови пројектног тима у развоју пројекта, а може се 
користити и при писању упутства за употребу.

\subsection{Референце}
\begin{enumerate}
	\item Опис пројектног задатка
	\item Упутство за писање спецификације сценарија случаја употребе
\end{enumerate}

\subsection{Отворена питања}
\begin{table}[h!]
\centering
	
	\begin{tabu}{ || X[l] | X[l] | X[l] | X[l] || }
	\hline
	\textbf{Верзија} & \textbf{Датум} & \textbf{Кратак опис} & \textbf{Аутор} \\
	\hline
	\hline
	1.0 & 10.03.2018. & Треба прецизирати која све подешавња постоје, за шта је неопходно
	прво прецизирати могућности саме игре. & Никола Јовановић \\
	\hline
	& & &\\
	\hline
	& & &\\
	\hline
	& & &\\
	\hline
	\end{tabu}
	\caption{Преглед отворених питања}
	\label{table:2}
		
\end{table}



\section{Сценарио креирања собе}

\subsection{Кратак опис}
Овде је представљен сценарио случаја коришћења система за креирање нове собе за игру.
Сценарио ће имати јединствени ток, при чему увек може бити прекинут, уколико корисник
изађе из истранице за креирање собе.

\subsection{Ток догађаја}
\begin{enumerate}
	\item Корисник уноси име собе
	\item Корисник бира да ли жели да игра против одређеног корисника, или жели да
	направи собу јавном, тако да било ко може да уђе у собу у улози играча, односно
	противника
	\item Корисник бира да ли жели да дозволи приступ соби посматрачима
	\item Корисник бира мапу за игру, или подешава насумично одабирање
	\item Корисник бира да ли жели да дозволи размену порука у соби
\end{enumerate}

\subsubsection{Корисник који је одабран за противника не постоји}
\begin{enumerate}[label=2.\arabic*]
	\item Систем исписује поруку да налог противника не постоји
	\item Систем враћа фокус на поље за одабир противника
	\item Повратак на корак 2 из главног сценарија
\end{enumerate}

\subsection{Посебни захтеви}
Нема.

\subsection{Предуслови}
Нема.

\subsection{Последице}
Преусмеравање играча на страницу за играње игре, где он чека противника да приступи соби како би игра започела.

