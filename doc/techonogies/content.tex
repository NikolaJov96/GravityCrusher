\section{Листа коришћених технологија}

\subsection{Израда прототипа веб апликације}
Приликом израде прототипа веб апликације, за израду веб страница коришћене су следеће технологије:
\begin{enumerate}
    \item HTML5
    \item CSS3
    \item Bootstrap 4
    \item JavaScript
\end{enumerate}

\subsection{Моделовање базе података}
Приликом моделовања базе подата коришћен је алат
erwin Data Modeler r9.8

\subsection{Систем за управљање базама података}
Приликом израде веб апликације коришћен је MySQL Server
систем за управљање базама података. Такође, коришћен
је и помоћни алат MySQL Workbench за рад са базом података.

\subsection{Моделовање веб апликације}
Приликом моделовања веб апликације коришћен је алат
StarUML v1. Примарно је коришћен StarUML v1 модул
\textit{Web Application Extension (WAE) Profile}.

\subsection{Имплементација \textit{frontend} дела веб апликације}
Приликом израде \textit{frontend} дела веб апликације коришћене
су следеће технологије:
\begin{enumerate}
    \item HTML5
    \item CSS3
    \item Bootstrap 4 (Bootstrap 4 \textit{Material Kit} алат за израду корисничког интерфејса)
    \item JavaScript
    \item WebGL -- za iscrtavanje grafike
\end{enumerate}

\subsection{Имплементација \textit{backend} дела веб апликације}
За израду \textit{backend} дела веб апликације коришћене су
следеће технологије:
\begin{enumerate}
    \item Node.js
        \begin{itemize}
            \item \textit{express} -- za uspostavljanje HTTP servera
            \item \textit{socket.io} -- za dinamičku komunikaciju sa serverom
            \item \textit{nodemailer} -- za komunikaciju sa korisnicima putem e-mail servisa
            \item \textit{mysql} -- za komunikaciju sa MySQL serverom
        \end{itemize}
    \item SQL
\end{enumerate}

\subsection{Израда документације}
За израду документације коришћен је LaTeX систем за
припрему докумената. Приликом писања документације
коришћен је текст едитор \texttt{vim}, а за генерисање
PDF докумената алат \texttt{pdflatex}.

